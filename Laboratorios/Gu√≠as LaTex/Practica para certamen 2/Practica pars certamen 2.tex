\documentclass[12pt]{article}
\usepackage[spanish]{babel} 
\usepackage[utf8]{inputenc}
\usepackage{amsmath} 
\usepackage{amsthm}  
\usepackage{amssymb}
\usepackage[shortlabels]{enumitem}
\begin{document}
\author{Moisés Amundarain}
\title{Ejercicio de Euler}
\date{23-05-2024}
\maketitle
Dada la función: 
\begin{equation*}
f(x)= e^{\sqrt{9-x^{2}}}
\end{equation*}

definir: 
\begin{enumerate}[a),leftmargin=2cm]	
\item Determinar si la función posee inversa. Justificando.
\item Definir las condiciones para que la función poseea inversa.
\end{enumerate}
\textbf{Solución:} 
\begin{enumerate}
	\item Lo primero, es calcular el Dom \emph{f}.
		\begin{align*}
			\text{Dom } f & = \{ x \in \mathbb{R}
 : 9-x^{2} \geq 0 \} \\ 
			& = \{ x \in \mathbb{R}
 : -3 \leq x \leq 3 \} \\ 
			& \iff [-3,3]	
			\end{align*}
	\item Lo segundo, es comprobrar la inyectividad. 
	
	Para esto, se muestra que $f(x)$, no es inyectiva, ya que para $f(-3) \land f(3)$, se llega a que: $f(-3)=1 \land f(3)=1 \land -3 \ne 3$. 
	\item Dado que que $f(x)$ no inyectiva, se procede a ``reparar la inyectividad".
	
	Dado $f(x) = e^{\sqrt{9-x^2}}$, mostrar que $f(a)=f(b)$
\newpage
	Entonces: 
		\begin{align*}
			e^{\sqrt{9-a^2}} &= e^{\sqrt{9-b^2}} & \mbox{/se aplica log. natural para sacar el número de euler}\\
			\sqrt{9-a^{2}} &= \sqrt{9-b^{2}} \\
			9-a^{2} &= 9-b^{2} \\
			a^2 &= b^2 \\
			|a| &= |b|\\		
		\end{align*}
\end{enumerate}
\end{document}